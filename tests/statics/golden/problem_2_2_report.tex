\documentclass[11pt,a4paper]{article}
\usepackage{amsmath}
\usepackage{amssymb}
\usepackage{booktabs}
\usepackage{longtable}
\usepackage{geometry}
\geometry{margin=1in}
\usepackage{hyperref}
\usepackage{enumitem}
\usepackage{graphicx}
\usepackage{siunitx}
\usepackage{accents}
\usepackage{changepage}
\usepackage{needspace}
\usepackage{tikz}
\usepackage{xcolor}
\usetikzlibrary{calc,arrows.meta}

% Vector notation: \vv{F} produces F with arrow over it
\newcommand{\vv}[1]{\vec{#1}}
% Magnitude notation: \magn{F} produces |F| with fixed-height bars
\newcommand{\magn}[1]{|#1|}

% TikZ styles for vector diagrams
\colorlet{vec_f1}{blue!80!black}
\colorlet{vec_f2}{red!80!black}
\colorlet{vec_fr}{green!60!black}
\colorlet{vec_translated}{gray!60}
\tikzset{
    vector/.style={-{Stealth[length=3mm,width=2mm]},very thick},
    vector_translated/.style={-{Stealth[length=2mm,width=1.5mm]},dashed,thin},
}

\title{Engineering Calculation Report: Problem 2-2}
\date{{{GENERATED_DATE}}}

\begin{document}
\maketitle

\section{Vector Diagrams}

\subsection*{Problem Setup}

\begin{center}
\begin{tikzpicture}[scale=1]
  % Define coordinates
  \coordinate (A) at (0.000,0.000);
  \coordinate (B) at (0.000,0.000);
  \coordinate (C) at (-5.956,-1.596);
  \coordinate (D) at (0.000,4.404);

  % Draw all four coordinate axes (as long as the longest vector)
  \draw[->,gray] (0.000,0.000) -- (6.166,0.000) node[right] {$+x$};
  \draw[->,gray] (0.000,0.000) -- (-6.166,0.000) node[left] {$-x$};
  \draw[->,gray] (0.000,0.000) -- (0.000,6.166) node[above] {$+y$};
  \draw[->,gray] (0.000,0.000) -- (0.000,-6.166) node[below] {$-y$};

  % Draw main vectors with arrows and magnitude labels (at tip)
  \draw[vector,vec_f1] (A) -- (B) node[below right,pos=1] {$\vv{F_1} = 960\,\text{N}$};
  \draw[vector,vec_f2] (A) -- (C) node[above left,pos=1] {$\vv{F_2} = 700\,\text{N}$};

  % Draw angle arcs (from reference axis to vector)
  \draw[vec_f1,thin] (0.000,0.000) ++(0:2.158) arc (0:45.21212697297036:2.158);
  \node[vec_f1,anchor=west] at (2.315,0.964) {$45^\circ$};
  \draw[vec_f2,thin] (0.000,0.000) ++(180:3.083) arc (180:195.0:3.083);
  \node[vec_f2,anchor=east] at (-3.404,-0.448) {$15^\circ$};

  % Draw origin point
  \fill (A) circle (2pt) node[below left] {$O$};
\end{tikzpicture}
\end{center}


\subsection*{Force Triangle (Parallelogram Law)}

\begin{center}
\begin{tikzpicture}[scale=1]
  % Define coordinates
  \coordinate (A) at (0.000,0.000);
  \coordinate (B) at (0.000,0.000);
  \coordinate (C) at (-5.956,-1.596);
  \coordinate (D) at (0.000,4.404);

  % Draw all four coordinate axes (as long as the longest vector)
  \draw[->,gray] (0.000,0.000) -- (6.166,0.000) node[right] {$+x$};
  \draw[->,gray] (0.000,0.000) -- (-6.166,0.000) node[left] {$-x$};
  \draw[->,gray] (0.000,0.000) -- (0.000,6.166) node[above] {$+y$};
  \draw[->,gray] (0.000,0.000) -- (0.000,-6.166) node[below] {$-y$};

  % Draw translated sides (parallelogram completion) - dashed
  \draw[vector_translated,vec_translated] (B) -- (D);
  \draw[vector_translated,vec_translated] (C) -- (D);

  % Draw main vectors with arrows and magnitude labels
  \draw[vector,vec_f1] (A) -- (B) node[pos=0.5,below right] {$\vv{F_1}$};
  \draw[vector,vec_f2] (A) -- (C) node[pos=0.5,above left] {$\vv{F_2}$};
  \draw[vector,vec_fr] (A) -- (D) node[pos=0.5,above right] {$\vv{F_R} = 500\,\text{N}$};

  % Draw interior angle arc (angle between F1 and F2)
  % The interior angle is 45°, drawn as an arc spanning that angle from F1.
  \draw[orange,thick] (0.000,0.000) ++(45.2:1.850) arc (45.2:45.2:1.850) node[midway,above] {$$};

  % Draw origin point
  \fill (A) circle (2pt) node[below left] {$O$};
\end{tikzpicture}
\end{center}


\section{Known Variables}

\begin{longtable}{lSSl}
\toprule
Vector & {$\magn{\vv{F}}$ (N)} & {$\theta$ (deg)} & Reference \\
\midrule
\endhead
$\vv{F_2}$ & 700.0 & 15.0 & $-x$ \\
$\vv{F_R}$ & 500.0 & 0.0 & $+y$ \\
\bottomrule
\end{longtable}

\section{Unknown Variables}

\begin{longtable}{lSSl}
\toprule
Vector & {$\magn{\vv{F}}$ (N)} & {$\theta$ (deg)} & Reference \\
\midrule
\endhead
$\vv{F_1}$ & ? & ? & $+x$ \\
\bottomrule
\end{longtable}

\section{Equations Used}

\begin{enumerate}
\item $\displaystyle \magn{\vv{F_1}}^2 = \magn{\vv{F_2}}^2 + \magn{\vv{F_R}}^2 - 2 \cdot \magn{\vv{F_2}} \cdot \magn{\vv{F_R}} \cdot \cos(\angle(\vv{F_2}, \vv{F_R}))$
\item $\displaystyle \frac{\sin(\angle(\vv{F_1}, \vv{F_R}))}{\magn{\vv{F_2}}} = \frac{\sin(\angle(\vv{F_1}, \vv{F_2}))}{\magn{\vv{F_1}}}$
\end{enumerate}

\section{Step-by-Step Solution}

\needspace{4\baselineskip}
\begin{samepage}
\vspace{0.2em}
\noindent\hspace{1em}\textbf{Step 1: Solve for $\angle(\vv{F_2}, \vv{F_R})$}
\vspace{0.1em}

\begin{adjustwidth}{2em}{}
\vspace{-0.8em}
\begin{flalign*}
\quad \angle(\vv{F_2}, \vv{F_R}) &= \magn{\angle(+x, \vv{F_2}) - \angle(+x, \vv{F_R})} && \\
\quad &= \magn{195^{\circ} - 90^{\circ}} && \\
\quad &= 105^{\circ} &&
\end{flalign*}
\vspace{-0.8em}
\end{adjustwidth}
\end{samepage}

\needspace{4\baselineskip}
\begin{samepage}
\vspace{0.2em}
\noindent\hspace{1em}\textbf{Step 2: Solve for $\magn{\vv{F_1}}$ using Eq 1}
\vspace{0.1em}

\begin{adjustwidth}{2em}{}
\vspace{-0.8em}
\begin{flalign*}
\quad \magn{\vv{F_1}} &= \sqrt{(700.0\ \text{N})^2 + (500.0\ \text{N})^2 - 2(700.0\ \text{N})(500.0\ \text{N})\cos(105.0^{\circ})} && \\
\quad &= 959.8\ \text{N} &&
\end{flalign*}
\vspace{-0.8em}
\end{adjustwidth}
\end{samepage}

\needspace{4\baselineskip}
\begin{samepage}
\vspace{0.2em}
\noindent\hspace{1em}\textbf{Step 3: Solve for $\angle(\vv{F_1}, \vv{F_R})$ using Eq 2}
\vspace{0.1em}

\begin{adjustwidth}{2em}{}
\vspace{-0.8em}
\begin{flalign*}
\quad \angle(\vv{F_1}, \vv{F_R}) &= \sin^{-1}(700.0\ \text{N} \cdot \frac{\sin(105.0^{\circ})}{959.8\ \text{N}}) && \\
\quad &= 44.8^{\circ} &&
\end{flalign*}
\vspace{-0.8em}
\end{adjustwidth}
\end{samepage}

\needspace{4\baselineskip}
\begin{samepage}
\vspace{0.2em}
\noindent\hspace{1em}\textbf{Step 4: Solve for $\angle(\vv{x}, \vv{F_1})$ with respect to +x}
\vspace{0.1em}

\begin{adjustwidth}{2em}{}
\vspace{-0.8em}
\begin{flalign*}
\quad \angle(\vv{x}, \vv{F_1}) &= \angle(\vv{x}, \vv{F_R}) - \angle(\vv{F_1}, \vv{F_R}) && \\
\quad &= 90.0^{\circ} - 44.8^{\circ} && \\
\quad &= 45.2^{\circ} &&
\end{flalign*}
\vspace{-0.8em}
\end{adjustwidth}
\end{samepage}

\section{Summary of Results}

\begin{longtable}{lSSl}
\toprule
Vector & {$\magn{\vv{F}}$ (N)} & {$\theta$ (deg)} & Reference \\
\midrule
\endhead
$\vv{F_1}$ & 959.8 & 45.2 & $+x$ \\
\bottomrule
\end{longtable}

\section*{Disclaimer}
\small
While every effort has been made to ensure the accuracy and reliability of the calculations provided, we do not guarantee that the information is complete, up-to-date, or suitable for any specific purpose. Users must independently verify the results and assume full responsibility for any decisions or actions taken based on its output. Use of this calculator is entirely at your own risk, and we expressly disclaim any liability for errors or omissions in the information provided.

\vspace{1em}
\noindent\textbf{Report Details:}
\begin{itemize}[nosep]
\item \textbf{Generated Date:} {{GENERATED_DATE}}
\item \textbf{Generated Using:} Qnty Library
\item \textbf{Version:} Beta (Independent verification required for production use)
\end{itemize}

\vspace{1em}
\noindent\textbf{Signatures:}
\begin{longtable}{llll}
\toprule
Role & Name & Signature & Date \\
\midrule
Calculated By & \rule{3cm}{0.4pt} & \rule{3cm}{0.4pt} & \rule{2cm}{0.4pt} \\
Reviewed By & \rule{3cm}{0.4pt} & \rule{3cm}{0.4pt} & \rule{2cm}{0.4pt} \\
Approved By & \rule{3cm}{0.4pt} & \rule{3cm}{0.4pt} & \rule{2cm}{0.4pt} \\
\bottomrule
\end{longtable}

\begin{center}
\textit{Report generated using qnty library}
\end{center}
\end{document}